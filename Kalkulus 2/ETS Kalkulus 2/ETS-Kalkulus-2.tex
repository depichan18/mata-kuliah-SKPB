\documentclass[12pt]{article}
\usepackage{amsmath, amssymb}
\usepackage{graphicx}
\usepackage{fancyhdr}
\usepackage{enumitem}
\usepackage{multicol}
\usepackage{geometry}
\geometry{margin=1in}

\pagestyle{fancy}
\fancyhf{}
\rhead{Devi Rosa Aprilla}
\lhead{Assignment}
\cfoot{\thepage}

\begin{document}

\begin{center}
    {Evaluasi Akhir Semester 2025} \\
    \textit{Kalkulus 2} \\
\end{center}

\vspace{0.5cm}

\begin{enumerate}
    \item Tentukan turunan pertama dari $y = \ln(\sinh(e^x))$.

    \item Diketahui fungsi $f(x) = \sin(2x)$.
    \begin{enumerate}
        \item Tentukan domain fungsi $f(x)$ agar mempunyai invers.
        \item Tentukan $f^{-1}(x)$.
        \item Buat grafik dari fungsi $f(x) = \sin(2x)$ dan grafik inversnya pada satu sistem koordinat.
    \end{enumerate}

    \item Selesaikan integral berikut:
    \[
        \int \frac{3\theta^2 - \theta + 1}{\theta^3 + \theta^2} \, d\theta 
    \]

    \item Hitung limit berikut:
    \[
        \lim_{x \to 0^+} x^{x}
    \]

    \item Diberikan daerah yang dibatasi oleh $y = \sqrt{x+2}$, $y = x$, dan $y = 0$:
    \begin{enumerate}
        \item Sketsa daerah tersebut.
        \item Dapatkan luasnya.
    \end{enumerate}
\end{enumerate}

\end{document}
